% resumo em português
\setlength{\absparsep}{18pt} % ajusta o espaçamento dos parágrafos do resumo
\begin{resumo}

Este trabalho analisa a eficiência e a escalabilidade de algoritmos paralelos presentes em benchmarks clássicos, utilizando implementações em OpenMP e linguagem C. Os algoritmos foram testados em diferentes configurações que variaram parâmetros de número de processadores, dimensão do problema e granularidade. Com base nas execuções, foram realizadas análises empíricas considerando os tempos de execução obtidos e métricas derivadas de \textit{speedup}, eficiência além do fator de \textit{overhead}.

Foi possível observar, após análise posterior dos resultados obtidos, comportamentos compartilhados entre certos algoritmos, correlações com características teóricas e possíveis problemas e gargalos enfrentados pela paralelização. Os resultados ajudaram a conceber um modelo analítico de escalabilidade que visa, de forma prática e teórica, amparar o cálculo da relação custo-benefício de aplicações paralelas.

 \textbf{Palavras-chave}: Computação Paralela. Escalabilidade. Eficiência. Benchmarks.
\end{resumo}