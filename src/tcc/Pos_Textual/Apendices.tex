\begin{apendicesenv}

\chapter{Gráficos de Speedup e Eficiência}

\section{Comparação entre Aplicações por Tamanho de Problema}

\begin{figure}[H]
\centering
\includegraphics[width=0.9\textwidth]{Graficos/comparison_small.png}
\caption{Comparação de speedup e eficiência - problema \textit{small}}
\label{fig:comparison_small_apendice}
\end{figure}

\begin{figure}[H]
\centering
\includegraphics[width=0.9\textwidth]{Graficos/comparison_medium.png}
\caption{Comparação de speedup e eficiência - problema \textit{medium}}
\label{fig:comparison_medium_apendice}
\end{figure}

\begin{figure}[H]
\centering
\includegraphics[width=0.9\textwidth]{Graficos/comparison_large.png}
\caption{Comparação de speedup e eficiência - problema \textit{large}}
\label{fig:comparison_large_apendice}
\end{figure}

\begin{figure}[H]
\centering
\includegraphics[width=0.9\textwidth]{Graficos/comparison_huge.png}
\caption{Comparação de speedup e eficiência - problema \textit{huge}}
\label{fig:comparison_huge_apendice}
\end{figure}

\begin{figure}[H]
\centering
\includegraphics[width=0.9\textwidth]{Graficos/comparison_extreme.png}
\caption{Comparação de speedup e eficiência - problema \textit{extreme}}
\label{fig:comparison_extreme_apendice}
\end{figure}

\section{Análise Individual por Aplicação}

\subsection{c\_mandel - Mandelbrot Set}

\begin{figure}[H]
\centering
\includegraphics[width=0.9\textwidth]{Graficos/c_mandel_speedup.png}
\caption{Speedup e eficiência de c\_mandel em diferentes tamanhos de problema}
\label{fig:mandel_speedup_apendice}
\end{figure}

\begin{figure}[H]
\centering
\includegraphics[width=0.9\textwidth]{Graficos/c_mandel_overhead.png}
\caption{Overhead relativo e fração serial de c\_mandel}
\label{fig:mandel_overhead_apendice}
\end{figure}

\subsection{c\_md - Molecular Dynamics}

\begin{figure}[H]
\centering
\includegraphics[width=0.9\textwidth]{Graficos/c_md_speedup.png}
\caption{Speedup e eficiência de c\_md em diferentes tamanhos de problema}
\label{fig:md_speedup_apendice}
\end{figure}

\begin{figure}[H]
\centering
\includegraphics[width=0.9\textwidth]{Graficos/c_md_overhead.png}
\caption{Overhead relativo e fração serial de c\_md}
\label{fig:md_overhead_apendice}
\end{figure}

\subsection{c\_pi - Monte Carlo Pi}

\begin{figure}[H]
\centering
\includegraphics[width=0.9\textwidth]{Graficos/c_pi_speedup.png}
\caption{Speedup e eficiência de c\_pi em diferentes tamanhos de problema}
\label{fig:pi_speedup_apendice}
\end{figure}

\begin{figure}[H]
\centering
\includegraphics[width=0.9\textwidth]{Graficos/c_pi_overhead.png}
\caption{Overhead relativo e fração serial de c\_pi}
\label{fig:pi_overhead_apendice}
\end{figure}

\subsection{c\_fft6 - Fast Fourier Transform (Radix-6)}

\begin{figure}[H]
\centering
\includegraphics[width=0.9\textwidth]{Graficos/c_fft6_speedup.png}
\caption{Speedup e eficiência de c\_fft6 em diferentes tamanhos de problema}
\label{fig:fft6_speedup_apendice}
\end{figure}

\begin{figure}[H]
\centering
\includegraphics[width=0.9\textwidth]{Graficos/c_fft6_overhead.png}
\caption{Overhead relativo e fração serial de c\_fft6}
\label{fig:fft6_overhead_apendice}
\end{figure}

\subsection{c\_fft - Fast Fourier Transform}

\begin{figure}[H]
\centering
\includegraphics[width=0.9\textwidth]{Graficos/c_fft_speedup.png}
\caption{Speedup e eficiência de c\_fft em diferentes tamanhos de problema}
\label{fig:fft_speedup_apendice}
\end{figure}

\begin{figure}[H]
\centering
\includegraphics[width=0.9\textwidth]{Graficos/c_fft_overhead.png}
\caption{Overhead relativo e fração serial de c\_fft}
\label{fig:fft_overhead_apendice}
\end{figure}

\subsection{c\_qsort - QuickSort}

\begin{figure}[H]
\centering
\includegraphics[width=0.9\textwidth]{Graficos/c_qsort_speedup.png}
\caption{Speedup e eficiência de c\_qsort em diferentes tamanhos de problema}
\label{fig:qsort_speedup_apendice}
\end{figure}

\begin{figure}[H]
\centering
\includegraphics[width=0.9\textwidth]{Graficos/c_qsort_overhead.png}
\caption{Overhead relativo e fração serial de c\_qsort}
\label{fig:qsort_overhead_apendice}
\end{figure}

\chapter{Análise de Overhead por Tamanho de Problema}

\begin{figure}[H]
\centering
\includegraphics[width=0.9\textwidth]{Graficos/overhead_small.png}
\caption{Overhead relativo de todas as aplicações - problema \textit{small}}
\label{fig:overhead_small_apendice}
\end{figure}

\begin{figure}[H]
\centering
\includegraphics[width=0.9\textwidth]{Graficos/overhead_medium.png}
\caption{Overhead relativo de todas as aplicações - problema \textit{medium}}
\label{fig:overhead_medium_apendice}
\end{figure}

\begin{figure}[H]
\centering
\includegraphics[width=0.9\textwidth]{Graficos/overhead_large.png}
\caption{Overhead relativo de todas as aplicações - problema \textit{large}}
\label{fig:overhead_large_apendice}
\end{figure}

\begin{figure}[H]
\centering
\includegraphics[width=0.9\textwidth]{Graficos/overhead_huge.png}
\caption{Overhead relativo de todas as aplicações - problema \textit{huge}}
\label{fig:overhead_huge_apendice}
\end{figure}

\begin{figure}[H]
\centering
\includegraphics[width=0.9\textwidth]{Graficos/overhead_extreme.png}
\caption{Overhead relativo de todas as aplicações - problema \textit{extreme}}
\label{fig:overhead_extreme_apendice}
\end{figure}

\chapter{Tabelas Completas de Resultados}

\section{Resultados Detalhados por Aplicação e Configuração}

\begin{table}[H]
\centering
\caption{Speedup observado para todas as aplicações (problema \textit{large})}
\label{tab:speedup_completo_large}
\small
\begin{tabular}{lrrrrrrr}
\toprule
\textbf{Aplicação} & \textbf{1T} & \textbf{2T} & \textbf{4T} & \textbf{8T} & \textbf{12T} & \textbf{16T} & \textbf{24T} \\
\midrule
c\_mandel & 1.00 & 1.98 & 3.93 & 7.45 & 10.89 & 14.32 & 18.98 \\
c\_md & 1.00 & 1.96 & 3.88 & 7.12 & 10.34 & 13.12 & 15.57 \\
c\_pi & 1.00 & 1.89 & 3.67 & 5.89 & 7.84 & 9.43 & 10.10 \\
c\_fft6 & 1.00 & 1.72 & 3.12 & 3.89 & 3.76 & 3.54 & 3.21 \\
c\_fft & 1.00 & 1.43 & 1.68 & 1.72 & 1.79 & 1.76 & 1.58 \\
c\_qsort & 1.00 & 1.74 & 1.52 & 1.45 & 1.32 & 1.21 & 1.12 \\
\bottomrule
\end{tabular}
\end{table}

\begin{table}[H]
\centering
\caption{Eficiência observada para todas as aplicações (problema \textit{large})}
\label{tab:eficiencia_completa_large}
\small
\begin{tabular}{lrrrrrrr}
\toprule
\textbf{Aplicação} & \textbf{1T} & \textbf{2T} & \textbf{4T} & \textbf{8T} & \textbf{12T} & \textbf{16T} & \textbf{24T} \\
\midrule
c\_mandel & 100.0\% & 99.0\% & 98.3\% & 93.1\% & 90.7\% & 89.5\% & 79.1\% \\
c\_md & 100.0\% & 98.0\% & 97.0\% & 89.0\% & 86.2\% & 82.0\% & 64.9\% \\
c\_pi & 100.0\% & 94.5\% & 91.8\% & 73.6\% & 65.3\% & 58.9\% & 42.1\% \\
c\_fft6 & 100.0\% & 86.0\% & 78.0\% & 48.6\% & 31.3\% & 22.1\% & 13.4\% \\
c\_fft & 100.0\% & 71.5\% & 42.0\% & 21.5\% & 14.9\% & 11.0\% & 6.6\% \\
c\_qsort & 100.0\% & 87.0\% & 38.0\% & 18.1\% & 11.0\% & 7.6\% & 4.7\% \\
\bottomrule
\end{tabular}
\end{table}

\begin{table}[H]
\centering
\caption{Tempos de execução serial ($T_1$) por tamanho de problema}
\label{tab:tempos_serial_completo}
\small
\begin{tabular}{lrrrrr}
\toprule
\textbf{Aplicação} & \textbf{small (s)} & \textbf{medium (s)} & \textbf{large (s)} & \textbf{huge (s)} & \textbf{extreme (s)} \\
\midrule
c\_mandel & 0.89 & 3.54 & 32.03 & 128.45 & 514.21 \\
c\_md & 0.67 & 2.78 & 22.14 & 88.92 & 356.18 \\
c\_pi & 0.12 & 0.51 & 5.17 & 20.76 & 83.12 \\
c\_fft6 & 0.34 & 1.89 & 8.94 & 35.87 & 143.61 \\
c\_fft & 0.28 & 1.56 & 7.48 & 30.12 & 120.54 \\
c\_qsort & 0.15 & 0.78 & 3.89 & 15.67 & 62.89 \\
\bottomrule
\end{tabular}
\end{table}

\begin{table}[H]
\centering
\caption{Overhead relativo $\phi = T_o/T_1$ para todas as aplicações (problema \textit{large}, 24 threads)}
\label{tab:overhead_phi_completo}
\small
\begin{tabular}{lrrrr}
\toprule
\textbf{Aplicação} & \textbf{$T_1$ (s)} & \textbf{$T_{24}$ (s)} & \textbf{$T_o$ (s)} & \textbf{$\phi$} \\
\midrule
c\_mandel & 32.03 & 1.687 & 8.46 & 0.26 \\
c\_md & 22.14 & 1.422 & 12.00 & 0.54 \\
c\_pi & 5.17 & 0.512 & 7.11 & 1.38 \\
c\_fft6 & 8.94 & 2.296 & 46.16 & 5.16 \\
c\_fft & 7.48 & 4.182 & 92.88 & 12.42 \\
c\_qsort & 3.89 & 2.314 & 51.65 & 13.28 \\
\bottomrule
\end{tabular}
\end{table}

\begin{table}[H]
\centering
\caption{Fração serial $\epsilon$ (Karp-Flatt) para todas as aplicações}
\label{tab:fracao_serial_completa}
\small
\begin{tabular}{lrrrrrrr}
\toprule
\textbf{Aplicação} & \textbf{2T} & \textbf{4T} & \textbf{8T} & \textbf{12T} & \textbf{16T} & \textbf{24T} & \textbf{Média} \\
\midrule
c\_mandel & 0.0051 & 0.0043 & 0.0092 & 0.0100 & 0.0094 & 0.0079 & 0.0077 \\
c\_md & 0.0102 & 0.0077 & 0.0137 & 0.0132 & 0.0142 & 0.0130 & 0.0120 \\
c\_pi & 0.0275 & 0.0223 & 0.0448 & 0.0623 & 0.0731 & 0.2687 & 0.0831 \\
c\_fft6 & 0.0814 & 0.0723 & 0.1645 & 0.2746 & 0.3520 & 0.2534 & 0.1997 \\
c\_fft & 0.1979 & 0.3452 & 0.4510 & 0.5023 & 0.5298 & 0.4339 & 0.4100 \\
c\_qsort & 0.0747 & 0.3947 & 0.5684 & 0.7348 & 0.9128 & 1.5569 & 0.7070 \\
\bottomrule
\end{tabular}
\end{table}

\begin{table}[H]
\centering
\caption{Score de escalabilidade $Es$ para todas as aplicações (problema \textit{large}, 24 threads)}
\label{tab:score_es_completo}
\small
\begin{tabular}{lrrrr}
\toprule
\textbf{Aplicação} & \textbf{Eficiência $E$} & \textbf{$\phi$} & \textbf{$\phi/\phi_{max}$} & \textbf{Score $Es$} \\
\midrule
c\_mandel & 0.791 & 0.26 & 0.020 & 0.775 \\
c\_md & 0.649 & 0.54 & 0.041 & 0.622 \\
c\_pi & 0.421 & 1.38 & 0.104 & 0.377 \\
c\_fft6 & 0.134 & 5.16 & 0.389 & 0.082 \\
c\_fft & 0.066 & 12.42 & 0.935 & 0.004 \\
c\_qsort & 0.047 & 13.28 & 1.000 & 0.000 \\
\bottomrule
\end{tabular}
\end{table}

\section{Coeficientes de Ajuste Polinomial}

\begin{table}[H]
\centering
\caption{Coeficientes do ajuste polinomial $S(p) = a_0 + a_1 p + a_2 p^2$}
\label{tab:coeficientes_polinomiais_completo}
\small
\begin{tabular}{lrrrr}
\toprule
\textbf{Aplicação} & \textbf{$a_0$} & \textbf{$a_1$} & \textbf{$a_2$} & \textbf{$R^2$} \\
\midrule
c\_mandel & -0.23 & 1.02 & -0.008 & 0.998 \\
c\_md & 0.15 & 0.95 & -0.012 & 0.995 \\
c\_pi & 0.45 & 0.82 & -0.018 & 0.991 \\
c\_fft6 & 0.67 & 0.51 & -0.025 & 0.987 \\
c\_fft & 0.89 & 0.21 & -0.035 & 0.923 \\
c\_qsort & 0.95 & 0.15 & -0.048 & 0.878 \\
\bottomrule
\end{tabular}
\end{table}

\chapter{Dados de Configuração Experimental}

\section{Especificações do Sistema}

\section{Parâmetros de Dimensão}

\begin{table}[H]
\centering
\caption{Parâmetros de dimensão utilizados para cada aplicação}
\label{tab:parametros_dimensao_completo}
\small
\begin{tabular}{lccccc}
\toprule
\textbf{Aplicação} & \textbf{small} & \textbf{medium} & \textbf{large} & \textbf{huge} & \textbf{extreme} \\
\midrule
c\_pi & 100k iter & 400k iter & 4M iter & 16M iter & 64M iter \\
c\_mandel & 200k pts & 800k pts & 3.2M pts & 12.8M pts & 51.2M pts \\
c\_fft & 1024 pts & 4096 pts & 16384 pts & 65536 pts & 262144 pts \\
c\_fft6 & 1296 pts & 7776 pts & 46656 pts & 279936 pts & 1679616 pts \\
c\_qsort & 1M elem & 4M elem & 16M elem & 64M elem & 256M elem \\
c\_md & 256 part & 512 part & 1024 part & 2048 part & 4096 part \\
\bottomrule
\end{tabular}
\end{table}

\end{apendicesenv}
